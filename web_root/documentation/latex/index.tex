\begin{DoxyAuthor}{Author}
Andre van Zuydam 
\end{DoxyAuthor}
\begin{DoxyVersion}{Version}
1.\+0.\+0 
\end{DoxyVersion}
\begin{DoxyCopyright}{Copyright}
Tina4
\end{DoxyCopyright}
\hypertarget{index_introduction}{}\section{Introduction to the Tina4\+Stack}\label{index_introduction}
The Tina4\+Stack started with a need to bring some uniformity to the development I was doing in P\+H\+P and especially where I was working with other developers on the same project. I have often wondered if I should start using one of the popular frameworks out there but have decided against it for various reasons. Instead I have decided to share with you my work bench so to speak where you can also benefit from the work that has gone into this tool.

\begin{DoxyNote}{Note}
Tina4 simply means \char`\"{}\+This is N\+O\+T another Framework!\char`\"{}
\end{DoxyNote}
\hypertarget{index_what_is_in_the_stack}{}\section{What is in the Stack?}\label{index_what_is_in_the_stack}
The Stack is built up from the following technologies which hopefully will get your development up and running in minutes. Incidentally we do not deploy any database engines for you and this should be done separately which is something we prefer.


\begin{DoxyItemize}
\item Nginx 
\item P\+H\+P 5.\+6 
\begin{DoxyEnumerate}
\item Curl is enabled 
\item G\+D2 is enabled 
\item My\+S\+Q\+Li is enabled 
\item S\+Q\+Lite3 is enabled 
\item X\+Debug is enabled on port 9000 
\item X\+Cache is enabled 
\item Exif is enabled 
\item File\+Info is enabled 
\item Mbstring is enabled 
\item Open\+S\+S\+L is enabled 
\item Soap is enabled 
\item X\+S\+L is enabled 
\end{DoxyEnumerate}
\item Doxygen -\/ both Windows \& Linux scripts -\/ you are reading generated documentation now! 
\item Swagger U\+I 
\item Selenium Client 
\item X\+Cache U\+I 
\end{DoxyItemize}

\begin{DoxyNote}{Note}
On my todo list are the Tina4 clients for Linux \& Mac\+O\+S
\end{DoxyNote}
\hypertarget{index_what_is_tina4}{}\section{What is Tina4}\label{index_what_is_tina4}
Tina4 is a collection of tools which will help you with your P\+H\+P development. In order to make the tools easy to remember and use we have given them names so that you can identify with each of the personalities when you are working. For example \hyperlink{classRuth}{Ruth} is responsible for Routing, \hyperlink{classMaggy}{Maggy} is responsible for migrations.

Now it may seem daunting to make use of tools you did not assist in building but you can use as little or as much of the tools as you want. After going over the core functionlity of each of the girls in the stack I will explain the basics of how everything is supposed to work.\hypertarget{index_tina}{}\section{Tina}\label{index_tina}
Tina is the C\+E\+O of the Tina4 Corporation, as anyone knows a successful company needs a strong leader. Her skills lie in the organization of each of the girls and making sure each person collaborates properly on the task at hand. Tina is passionate about providing solutions to her customers and it is very important to her that each peson is doing the job they were assigned. In a nutshell we have the following\+:


\begin{DoxyItemize}
\item Routing  
\item Migrations  
\item Database Abstraction  
\item Code Simplification  
\item Templating  
\item U\+I Testing with Selenium  
\item W\+Y\+I\+W\+Y\+G Report Generation  
\item Email Handling  
\item Object Abstraction  
\end{DoxyItemize}\hypertarget{index_ruth}{}\section{Ruth}\label{index_ruth}
\hyperlink{classRuth}{Ruth} as we have said before is responsible for routing. Routing as you may be familiar with is the ability to handle U\+R\+L requests to your website. You may be quite comfortable with writing routing tables in Apache or N\+G\+I\+N\+X. If you decide to use \hyperlink{classRuth}{Ruth} you immediately have an application which will run on N\+G\+I\+N\+X and Apache equally well. \hyperlink{classRuth}{Ruth} also handles security and sessions which are explained in her section.

A quick example of \hyperlink{classRuth}{Ruth} in action, you would get this link at \href{http://localhost:12345/testing}{\tt http\+://localhost\+:12345/testing} when developing. The output of this route would be {\itshape Hello} {\itshape Tina4!} \hypertarget{index_ruth_example}{}\subsection{Ruth Example}\label{index_ruth_example}

\begin{DoxyCode}
\textcolor{comment}{//Adding a GET route}
\hyperlink{classRuth_ad13bc87f60f8b74efd4c784fa2c49288}{Ruth::addRoute} (\hyperlink{Ruth_8php_acc73cd34b6fcffd15a4fdd050e0a86d8}{RUTH\_GET}, \textcolor{stringliteral}{"/testing"}, \textcolor{keyword}{function} () \{
     echo \textcolor{stringliteral}{"Hello Tina4!"};
\});
\end{DoxyCode}


\begin{DoxyNote}{Note}
Simply adding a P\+H\+P file under the routes folder and adding the above code is sufficient for \hyperlink{classRuth}{Ruth} to start working.
\end{DoxyNote}
\begin{DoxyRemark}{Remarks}
The setup of the web server routes all U\+R\+L requests to the \hyperlink{tina4_8php}{tina4.\+php} file which in turn relays them to \hyperlink{classRuth}{Ruth}. If your webserver has not been setup the stack will attempt to give you a configuration for Nginx or Apache.
\end{DoxyRemark}
\hypertarget{index_cody}{}\section{Cody}\label{index_cody}
\hyperlink{classCody}{Cody} was written to automate things that may take up a lot of repetition. At this point in time \hyperlink{classCody}{Cody} helps automate Bootstrap functionlity. \hyperlink{classCody}{Cody} has a very neat Ajax handler which automatically reads form variables and passes them to the U\+R\+L you specify. The Ajax handler of \hyperlink{classCody}{Cody} can also handle file uploading via Ajax and when used with \hyperlink{classRuth}{Ruth} and \hyperlink{classDebby}{Debby} you have very little coding to do.\hypertarget{index_cody_example}{}\subsection{Example of Bootstrap Input}\label{index_cody_example}
The following code will create a properly formatted Bootstrap input with all the correct classes. You can see this in action in \hyperlink{classMaggy}{Maggy} \href{http://localhost:12345/maggy/create}{\tt http\+://localhost\+:12345/maggy/create} 
\begin{DoxyCode}
echo (\textcolor{keyword}{new} \hyperlink{classCody}{Cody}())->bootStrapInput(\textcolor{stringliteral}{"firstName"}, \textcolor{stringliteral}{"First Name"}, \textcolor{stringliteral}{"Full Names"}, $defaultValue = \textcolor{stringliteral}{""}, \textcolor{stringliteral}{"text"});
\end{DoxyCode}
\hypertarget{index_code_example_ajax}{}\subsection{Example of Ajax Handler}\label{index_code_example_ajax}
The following code will create a Javascript function called {\itshape call\+Ajax} which can be used to run Ajax commands to your routing 
\begin{DoxyCode}
echo (\textcolor{keyword}{new} \hyperlink{classCody}{Cody}())->ajaxHandler();
\end{DoxyCode}
 